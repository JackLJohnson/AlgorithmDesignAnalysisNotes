\ifx\PREAMBLE\undefined
\documentclass{report}
\usepackage[format = hang, font = bf]{caption}
% The following is needed in order to make the code compatible
% with both latex/dvips and pdflatex. Added for using UML generated by MetaUML.
\ifx\pdftexversion\undefined
\usepackage[dvips]{graphicx}
\else
\usepackage[pdftex]{graphicx}
\DeclareGraphicsRule{*}{mps}{*}{}
\fi
\usepackage{array}
\usepackage{amsmath}
\usepackage{amsthm}
\usepackage{mathtools}
\usepackage{boxedminipage}
\usepackage{listings}
\usepackage{makecell}%diagonal line in table
\usepackage{float}%allowing forceful figure[H]
\usepackage{xcolor}
\usepackage{amsfonts}%allowing \mathbb{R}
\usepackage{amssymb}
\usepackage{alltt}
\usepackage{algorithmicx}
\usepackage[chapter]{algorithm} 
%chapter option ensures that algorithms are numbered within each chapter rather than in the whole article
\usepackage[noend]{algpseudocode} %If end if, end procdeure, etc is expected to appear, remove the noend option
\usepackage{xspace}
\usepackage{color}
\usepackage{url}
\def\UrlBreaks{\do\A\do\B\do\C\do\D\do\E\do\F\do\G\do\H\do\I\do\J\do\K\do\L\do\M\do\N\do\O\do\P\do\Q\do\R\do\S\do\T\do\U\do\V\do\W\do\X\do\Y\do\Z\do\[\do\\\do\]\do\^\do\_\do\`\do\a\do\b\do\c\do\d\do\e\do\f\do\g\do\h\do\i\do\j\do\k\do\l\do\m\do\n\do\o\do\p\do\q\do\r\do\s\do\t\do\u\do\v\do\w\do\x\do\y\do\z\do\0\do\1\do\2\do\3\do\4\do\5\do\6\do\7\do\8\do\9\do\.\do\@\do\\\do\/\do\!\do\_\do\|\do\;\do\>\do\]\do\)\do\,\do\?\do\'\do+\do\=\do\#\do\-}
\usepackage[breaklinks = true]{hyperref}
\lstset{
language = C++, 
showspaces = false,
breaklines = true, 
tabsize = 2, 
numbers = left, 
extendedchars = false, 
basicstyle = {\ttfamily \footnotesize}, 
keywordstyle=\color{blue!70}, 
commentstyle=\color{gray}, 
frame=shadowbox, 
rulesepcolor=\color{red!20!green!20!blue!20}, 
numberstyle={\color[RGB]{0,192,192}}, 
moredelim=[is][\underbar]{_}{_}
}
\mathchardef\myhyphen="2D
% switch-case environment definitions
\algblock{switch}{endswitch} 
\algblock{case}{endcase}
%\algrenewtext{endswitch}{\textbf{end switch}} %If end switch is expected to appear, uncomment this line.
\algtext*{endswitch} % Make end switch disappear
\algtext*{endcase}
\algnewcommand\algorithmicinput{\textbf{input:}}
\algnewcommand\Input{\item[\algorithmicinput]}
\algnewcommand\algorithmicoutput{\textbf{output:}}
\algnewcommand\Output{\item[\algorithmicoutput]}
\allowdisplaybreaks
\newtheorem{theorem}{Theorem}[chapter]
\newtheorem{corollary}[theorem]{Corollary}
\newtheorem{lemma}[theorem]{Lemma}
\newtheorem{definition}{Definition}[chapter]
\begin{document}
\fi
\chapter{Graph Algorithms}
\textbf{Graphs represent pairwise relationships amongst a set of objects}. The objects are called vertices or nodes. The relationships are called edges or arcs, each connecting a pair of vertices. An edge can be directed or undirected. The set of vertices and the set of edges are noted respectively as $V$ and $E$. Graph is a concept heavily used in reality. Road networks, the web, social networks, precedence constraints are all examples of graphs.

A connected graph composed of $n$ vertices with no parallel edges has at least $n-1$ and at most $n(n-1)/2$ edges. Let $m$ represent the number of edges. In most applications, $m$ is $\Omega(n)$ and $O(n^2)$. If $m$ is $O(n)$ or close to it, the graph is called a sparse graph, while if $m$ is closer to $O(n^2)$, it's called a dense graph. Yet their delimitation is not strictly clear. 
\section{Representation}
\subsection{Adjacent Matrix}
An undirected graph $G$ with $n$ vertices and no parallel edges can be represented by an $n\times n$ 0-1 matrix $A$. $A_{ij}=1$ when and only when $G$ has an $i-j$ edge. Variants of this representation can easily accommodate parallel edges, weighted edges: just let $A_{ij}$ represent the number of parallel edges or the weight of the edge. For directed graphs, $i\rightarrow j$ can be represented by $A_{ij}=1$ and $A_{ji}=-1$.

Adjacent matrix representation requires $\Theta(n^2)$ space. For a dense graph this is fine, but for a sparse graph it is wasteful. 
\subsection{Adjacent Lists}
The adjacent lists representation is composed of 4 ingredients:
\begin{itemize}
\item Array/List of vertices. $\Theta(n)$ space.
\item Array/List of edges. $\Theta(m)$ space.
\item Each edge points to its end points. $\Theta(m)$ space.
\item Each vertex points to edges incident on it. $\Theta(m)$ space.
\end{itemize}
Adjacent lists representation requires $\Theta(n+m)$ space. 

The choice between the two representations depends on the density of the graph and operations to take. We will mainly use adjacent lists in this chapter.
\section{Minimum Cut}
\subsection{Definition}
\begin{definition}
A cut of a graph ($V,E$) is a partition of $V$ into two non-empty sets $A$ and $B$.
\end{definition}
A graph with $n$ vertices has $2^n-2$ possible cuts. 
\begin{definition}
The crossing edges of a cut($A,B$) are those with 
\begin{itemize}
\item one endpoint in $A$ and the other in $B$, for undirected graphs;
\item tail in $A$ and head in $B$, for directed graphs.
\end{itemize}
\end{definition}
We will try to solve the minimum cut problem:
\begin{description}
\item[Input]An undirected graph $G=(V,E)$ in which parallel edges are allowed.
\item[Output]A cut $(A,B)$ with minimum number of crossing edges.
\end{description}
A lot of problems in reality can be reduced to a minimum cut problems:
\begin{itemize}
\item Identify weakness point of physical networks;
\item Community detection in social networks;
\item Image segmentation.
\end{itemize}
\subsection{Random Contraction Algorithm}
\begin{algorithm}[ht]
\caption{Random Contraction}\label{randomcontraction}
\begin{algorithmic}[1]
\Input\Statex{An undirected graph $G=(V,E)$ in which parallel edges are allowed.}
\Output\Statex{A cut $(A,B)$ with minimum number of crossing edges}
\While{There are more than 2 vertices left}
\State{Pick a remaining edge $(u,v)$ randomly}
\State{Merge(or contract) $u,v$ into a single vertex}
\State{Remove self-loops}
\EndWhile
\State{Return cut represented by 2 final vertices}
\end{algorithmic}
\end{algorithm}
\subsection{Number of Minimum Cuts}
A graph can have multiple minimum cuts. For example, a tree with $n$ vertices has $n-1$ minimum cuts. We would like to find the largest number of minimum cuts that a graph with $n$ vertices can have.
\begin{theorem}
A graph with $n$ vertices can have at most $\binom{n}{2}$ minimum cuts.
\end{theorem}
\begin{proof}
Consider the graph in which the $n$ vertices form a circle. Removing any two edges results in a minimum cut. Thus in total it has $\binom{n}{2}$ minimum cuts. The rest of the proof aims at proving that a graph with $n$ vertices cannot have more minimum cuts.

Recall that the probability for Algorithm \ref{randomcontraction} to return a specific minimum cut is bigger than $\frac{2}{n(n-1)}$. Suppose there are $k$ minimum cuts. The events ``Algorithm \ref{randomcontraction} returns minimum cut $C_i$'' and  ``Algorithm \ref{randomcontraction} returns minimum cut $C_j$'' are disjoint events when $i\neq j$. Thus we have 
$$1\geq P(\text{return a minimum cut})\geq\frac{2k}{n(n-1)}.$$
Therefore, 
$$k\leq \frac{n(n-1)}{2}=\binom{n}{2}.$$
\end{proof}
\section{Breadth First Search}
Graph search is widely used for various purposes:
\begin{itemize}
\item Check if a network is connected;
\item Find shortest paths, e.g. for driving navigation, or formulating a plan;
\item Calculate connected components.
\item $\dots$
\end{itemize}
We will introduce a few fast algorithms based on graph search. Graph search usually starts from a source vertex. When searching the graph, we want to find everything that is findable, i.e. every vertex reachable from the source via a path. Moreover, we never explore anything twice. In terms of running time, our goal is $O(m+n).$ 
\subsection{Algorithm}
BFS explores the nodes of a graph in layers. Nodes with the same distances from the source are in the same layer. It can be used to compute shortest paths of graphs, and to compute connected components of undirected graphs. It guarantees $O(m+n)$ running time. The general pattern of BFS is shown in Algorithm \ref{bfs}.
\begin{algorithm}[ht]
\caption{Breadth First Search(BFS)}\label{bfs}
\begin{algorithmic}[1]
\Input\Statex{Graph $G$ with all vertices unexplored}
\Statex{Source vertex $s$}
\Output\Statex{$G$ with all vertices reachable from $s$ explored.}
\State{Mark $s$ as explored.}
\State{Let $Q$ = queue initialized with $s$}
\While{$Q\neq\emptyset$}
\State{Remove first element $v$ of $Q$}
\For{each edge $(v,w)$}
\If{$w$ is unexplored}
\State{Mark $w$ as explored}
\State{Add $w$ to $Q$}
\EndIf
\EndFor
\EndWhile 
\end{algorithmic}
\end{algorithm}

At the end of BFS, the fact that a node $v$ is explored means the existence of a path from $s$ to $v$. 
\subsection{Shortest Path}
Algorithm \ref{shortestpath} calculates the shortest path from $s$ to any vertex reachable from $s$.
\begin{algorithm}[ht]
\caption{Shortest Path (BFS)}\label{shortestpath}
\begin{algorithmic}[1]
\Input\Statex{Graph $G$ with all vertices unexplored}
\Statex{Source vertex $s$}
\Output\Statex{$dist(v)$ for any vertex $v$, i.e. min number of edges on a path from $s$ to $v$}
\State{Initialize $dist(v)=\left\{\begin{array}{rl}0,&if(v==s)\\+\infty,&if(v\neq s)\end{array}\right.$}
\State{Mark $s$ as explored.}
\State{Let $Q$ = queue initialized with $s$}
\While{$Q\neq\emptyset$}
\State{Remove first element $v$ of $Q$}
\For{each edge $(v,w)$}
\If{$w$ is unexplored}
\State{Mark $w$ as explored}
\State{$dist(w)=dist(v)+1$}
\State{Add $w$ to $Q$}
\EndIf
\EndFor
\EndWhile 
\end{algorithmic}
\end{algorithm}

After the algorithm terminates, $dist(v)=i$ means that $v$ is in the $i^{th}$ layer and that the shortest $s-v$ path has $i$ edges.
\subsection{Undirected Connectivity}
\section{Depth First Search}
\ifx\PREAMBLE\undefined
\end{document}
\fi